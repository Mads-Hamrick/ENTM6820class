% Options for packages loaded elsewhere
\PassOptionsToPackage{unicode}{hyperref}
\PassOptionsToPackage{hyphens}{url}
%
\documentclass[
  12pt,
]{article}
\usepackage{amsmath,amssymb}
\usepackage{iftex}
\ifPDFTeX
  \usepackage[T1]{fontenc}
  \usepackage[utf8]{inputenc}
  \usepackage{textcomp} % provide euro and other symbols
\else % if luatex or xetex
  \usepackage{unicode-math} % this also loads fontspec
  \defaultfontfeatures{Scale=MatchLowercase}
  \defaultfontfeatures[\rmfamily]{Ligatures=TeX,Scale=1}
\fi
\usepackage{lmodern}
\ifPDFTeX\else
  % xetex/luatex font selection
\fi
% Use upquote if available, for straight quotes in verbatim environments
\IfFileExists{upquote.sty}{\usepackage{upquote}}{}
\IfFileExists{microtype.sty}{% use microtype if available
  \usepackage[]{microtype}
  \UseMicrotypeSet[protrusion]{basicmath} % disable protrusion for tt fonts
}{}
\makeatletter
\@ifundefined{KOMAClassName}{% if non-KOMA class
  \IfFileExists{parskip.sty}{%
    \usepackage{parskip}
  }{% else
    \setlength{\parindent}{0pt}
    \setlength{\parskip}{6pt plus 2pt minus 1pt}}
}{% if KOMA class
  \KOMAoptions{parskip=half}}
\makeatother
\usepackage{xcolor}
\usepackage[margin=1in]{geometry}
\usepackage{color}
\usepackage{fancyvrb}
\newcommand{\VerbBar}{|}
\newcommand{\VERB}{\Verb[commandchars=\\\{\}]}
\DefineVerbatimEnvironment{Highlighting}{Verbatim}{commandchars=\\\{\}}
% Add ',fontsize=\small' for more characters per line
\usepackage{framed}
\definecolor{shadecolor}{RGB}{248,248,248}
\newenvironment{Shaded}{\begin{snugshade}}{\end{snugshade}}
\newcommand{\AlertTok}[1]{\textcolor[rgb]{0.94,0.16,0.16}{#1}}
\newcommand{\AnnotationTok}[1]{\textcolor[rgb]{0.56,0.35,0.01}{\textbf{\textit{#1}}}}
\newcommand{\AttributeTok}[1]{\textcolor[rgb]{0.13,0.29,0.53}{#1}}
\newcommand{\BaseNTok}[1]{\textcolor[rgb]{0.00,0.00,0.81}{#1}}
\newcommand{\BuiltInTok}[1]{#1}
\newcommand{\CharTok}[1]{\textcolor[rgb]{0.31,0.60,0.02}{#1}}
\newcommand{\CommentTok}[1]{\textcolor[rgb]{0.56,0.35,0.01}{\textit{#1}}}
\newcommand{\CommentVarTok}[1]{\textcolor[rgb]{0.56,0.35,0.01}{\textbf{\textit{#1}}}}
\newcommand{\ConstantTok}[1]{\textcolor[rgb]{0.56,0.35,0.01}{#1}}
\newcommand{\ControlFlowTok}[1]{\textcolor[rgb]{0.13,0.29,0.53}{\textbf{#1}}}
\newcommand{\DataTypeTok}[1]{\textcolor[rgb]{0.13,0.29,0.53}{#1}}
\newcommand{\DecValTok}[1]{\textcolor[rgb]{0.00,0.00,0.81}{#1}}
\newcommand{\DocumentationTok}[1]{\textcolor[rgb]{0.56,0.35,0.01}{\textbf{\textit{#1}}}}
\newcommand{\ErrorTok}[1]{\textcolor[rgb]{0.64,0.00,0.00}{\textbf{#1}}}
\newcommand{\ExtensionTok}[1]{#1}
\newcommand{\FloatTok}[1]{\textcolor[rgb]{0.00,0.00,0.81}{#1}}
\newcommand{\FunctionTok}[1]{\textcolor[rgb]{0.13,0.29,0.53}{\textbf{#1}}}
\newcommand{\ImportTok}[1]{#1}
\newcommand{\InformationTok}[1]{\textcolor[rgb]{0.56,0.35,0.01}{\textbf{\textit{#1}}}}
\newcommand{\KeywordTok}[1]{\textcolor[rgb]{0.13,0.29,0.53}{\textbf{#1}}}
\newcommand{\NormalTok}[1]{#1}
\newcommand{\OperatorTok}[1]{\textcolor[rgb]{0.81,0.36,0.00}{\textbf{#1}}}
\newcommand{\OtherTok}[1]{\textcolor[rgb]{0.56,0.35,0.01}{#1}}
\newcommand{\PreprocessorTok}[1]{\textcolor[rgb]{0.56,0.35,0.01}{\textit{#1}}}
\newcommand{\RegionMarkerTok}[1]{#1}
\newcommand{\SpecialCharTok}[1]{\textcolor[rgb]{0.81,0.36,0.00}{\textbf{#1}}}
\newcommand{\SpecialStringTok}[1]{\textcolor[rgb]{0.31,0.60,0.02}{#1}}
\newcommand{\StringTok}[1]{\textcolor[rgb]{0.31,0.60,0.02}{#1}}
\newcommand{\VariableTok}[1]{\textcolor[rgb]{0.00,0.00,0.00}{#1}}
\newcommand{\VerbatimStringTok}[1]{\textcolor[rgb]{0.31,0.60,0.02}{#1}}
\newcommand{\WarningTok}[1]{\textcolor[rgb]{0.56,0.35,0.01}{\textbf{\textit{#1}}}}
\usepackage{graphicx}
\makeatletter
\def\maxwidth{\ifdim\Gin@nat@width>\linewidth\linewidth\else\Gin@nat@width\fi}
\def\maxheight{\ifdim\Gin@nat@height>\textheight\textheight\else\Gin@nat@height\fi}
\makeatother
% Scale images if necessary, so that they will not overflow the page
% margins by default, and it is still possible to overwrite the defaults
% using explicit options in \includegraphics[width, height, ...]{}
\setkeys{Gin}{width=\maxwidth,height=\maxheight,keepaspectratio}
% Set default figure placement to htbp
\makeatletter
\def\fps@figure{htbp}
\makeatother
\setlength{\emergencystretch}{3em} % prevent overfull lines
\providecommand{\tightlist}{%
  \setlength{\itemsep}{0pt}\setlength{\parskip}{0pt}}
\setcounter{secnumdepth}{-\maxdimen} % remove section numbering
\ifLuaTeX
  \usepackage{selnolig}  % disable illegal ligatures
\fi
\usepackage{bookmark}
\IfFileExists{xurl.sty}{\usepackage{xurl}}{} % add URL line breaks if available
\urlstyle{same}
\hypersetup{
  pdftitle={2025\_4\_3\_CodingChallengeSix\_mer0127},
  pdfauthor={Madeline Redd},
  hidelinks,
  pdfcreator={LaTeX via pandoc}}

\title{2025\_4\_3\_CodingChallengeSix\_mer0127}
\author{Madeline Redd}
\date{2025 - 04 - 06}

\begin{document}
\maketitle

\subsection{Question One}\label{question-one}

Regarding reproducibility, what is the main point of writing your own
functions and iterations?

Answer: Creating functions allows others opportunity to see each step of
the calculation and will reduce human error if the operations had to be
constantly retyped. It could benefit someone when reviewing the code and
if needed to build on the code. For iterations, it allows replications
to be accurate and efficiently if preformed exactly the same, but also
allows other to test different parameters.

\subsection{Question Two}\label{question-two}

In your own words, describe how to write a function and a for loop in R
and how they work. Give me specifics like syntax, where to write code,
and how the results are returned.

Answer:

Example of Creating a Function Code --\textgreater{}

new\_function \textless- function(argument\_one, argument\_two) \{
answer \textless- argument\_one * (argument\_two + 87) \#Operations to
perform with arguments to produce results return(answer)\\
\}

new\_function(45, 3)

Example of FOR LOOP Code --\textgreater{}

iteration\_length \textless- lage-fage+1 \# Define the iteration length

\#Necessary Parameters for the For Loop \#Best for stable coefficients
or numbers that might need to be adjusted and help reduce human error

M=0.4 \#Natural Mortality survship0=numeric(iteration\_length) \#Empty
numeric vector for data survship0{[}1{]}=1 \#survivorship at age 1

\#a is the loop variable \#the 2:iteration\_length: the sequence it will
iterate over

for (a in 2:iteration\_length)\{
survship0{[}a{]}=survship0{[}a-1{]}*exp(-M) \#Writing calculation code
could be built into almost anything.

\}

print(survship0) \#print data

\begin{Shaded}
\begin{Highlighting}[]
\CommentTok{\#Full code for Mortality with Fishing Pressure on Atlantic Tarpon}
\CommentTok{\#Parameters for for loop}
\NormalTok{fage}\OtherTok{=}\DecValTok{1} \CommentTok{\#first age}
\NormalTok{lage}\OtherTok{=}\DecValTok{8} \CommentTok{\#last age}
\NormalTok{linf}\OtherTok{=}\DecValTok{360} \CommentTok{\#Maximum length }
\NormalTok{k}\OtherTok{=}\FloatTok{0.45} \CommentTok{\# growth coefficient}
\NormalTok{t0}\OtherTok{=}\DecValTok{0} \CommentTok{\#age 0}
\NormalTok{cv\_tl}\OtherTok{=}\FloatTok{0.1} \CommentTok{\#variation in total length}
\NormalTok{M}\OtherTok{=}\FloatTok{0.4} \CommentTok{\#Natural mortality }
\NormalTok{l50\_mat}\OtherTok{=}\DecValTok{225} \CommentTok{\#length at 50\% pop of maturity}
\NormalTok{h\_mat}\OtherTok{=}\FloatTok{0.05} \CommentTok{\#steepness of maturity }
\NormalTok{a\_wt}\OtherTok{=}\FloatTok{2e{-}5} \CommentTok{\#length{-}weight}
\NormalTok{b\_wt}\OtherTok{=}\DecValTok{3} \CommentTok{\#length{-}weight}
\NormalTok{alpha\_hat}\OtherTok{=}\DecValTok{8} \CommentTok{\#recruitment}
\NormalTok{R0}\OtherTok{=}\DecValTok{1} \CommentTok{\#initial recruitment }
\NormalTok{tmax}\OtherTok{=}\DecValTok{100} \CommentTok{\#max years}
\NormalTok{F\_full}\OtherTok{=}\FloatTok{0.5} \CommentTok{\#Fishing Mortality}
\NormalTok{MLL}\OtherTok{=}\DecValTok{250} \CommentTok{\#Minimum legal length }

\NormalTok{age}\OtherTok{=}\NormalTok{fage}\SpecialCharTok{:}\NormalTok{lage}
\NormalTok{n\_ages}\OtherTok{=}\NormalTok{lage}\SpecialCharTok{{-}}\NormalTok{fage}\SpecialCharTok{+}\DecValTok{1}
\NormalTok{length}\OtherTok{=}\NormalTok{linf}\SpecialCharTok{*}\NormalTok{(}\DecValTok{1}\SpecialCharTok{{-}}\FunctionTok{exp}\NormalTok{(}\SpecialCharTok{{-}}\NormalTok{k}\SpecialCharTok{*}\NormalTok{(age}\SpecialCharTok{{-}}\NormalTok{t0)))}
\NormalTok{weight}\OtherTok{=}\NormalTok{a\_wt}\SpecialCharTok{*}\NormalTok{length}\SpecialCharTok{\^{}}\NormalTok{b\_wt}
\NormalTok{maturity}\OtherTok{=}\DecValTok{1}\SpecialCharTok{/}\NormalTok{(}\DecValTok{1}\SpecialCharTok{+}\FunctionTok{exp}\NormalTok{(}\SpecialCharTok{{-}}\NormalTok{h\_mat}\SpecialCharTok{*}\NormalTok{(length}\SpecialCharTok{{-}}\NormalTok{l50\_mat)))}
\NormalTok{fecundity}\OtherTok{=}\NormalTok{weight}\SpecialCharTok{*}\NormalTok{maturity}
\NormalTok{sd\_length}\OtherTok{=}\NormalTok{cv\_tl}\SpecialCharTok{*}\NormalTok{length}
\NormalTok{vuln}\OtherTok{=}\DecValTok{1}\SpecialCharTok{{-}}\FunctionTok{pnorm}\NormalTok{(MLL,length,sd\_length)}

\NormalTok{survship0}\OtherTok{=}\FunctionTok{numeric}\NormalTok{(n\_ages)}
\NormalTok{survship0[}\DecValTok{1}\NormalTok{]}\OtherTok{=}\DecValTok{1}

\CommentTok{\#Survivorship For Loop}
\ControlFlowTok{for}\NormalTok{ (a }\ControlFlowTok{in} \DecValTok{2}\SpecialCharTok{:}\NormalTok{n\_ages)\{}
\NormalTok{  survship0[a]}\OtherTok{=}\NormalTok{survship0[a}\DecValTok{{-}1}\NormalTok{]}\SpecialCharTok{*}\FunctionTok{exp}\NormalTok{(}\SpecialCharTok{{-}}\NormalTok{M)}
\NormalTok{\}}
\NormalTok{survship0[lage]}\OtherTok{=}\NormalTok{survship0[lage]}\SpecialCharTok{/}\NormalTok{(}\DecValTok{1}\SpecialCharTok{{-}}\FunctionTok{exp}\NormalTok{(}\SpecialCharTok{{-}}\NormalTok{M))}

\NormalTok{ssb0}\OtherTok{=}\FunctionTok{sum}\NormalTok{(survship0}\SpecialCharTok{*}\NormalTok{fecundity)}
\NormalTok{alpha}\OtherTok{=}\NormalTok{alpha\_hat}\SpecialCharTok{/}\NormalTok{ssb0}
\NormalTok{beta}\OtherTok{=}\FunctionTok{log}\NormalTok{(alpha}\SpecialCharTok{*}\NormalTok{ssb0)}\SpecialCharTok{/}\NormalTok{(R0}\SpecialCharTok{*}\NormalTok{ssb0)}


\NormalTok{N\_a}\OtherTok{=}\FunctionTok{matrix}\NormalTok{(}\ConstantTok{NA}\NormalTok{,}\AttributeTok{nrow=}\NormalTok{tmax,}\AttributeTok{ncol=}\NormalTok{n\_ages) }\CommentTok{\#Blank matrix for values to be imported into}
\NormalTok{F\_a}\OtherTok{=}\NormalTok{N\_a}
\NormalTok{Z\_a}\OtherTok{=}\NormalTok{N\_a}
\NormalTok{U\_a}\OtherTok{=}\NormalTok{N\_a}
\NormalTok{yield\_t}\OtherTok{=}\FunctionTok{numeric}\NormalTok{(tmax)}
\NormalTok{ssb\_t}\OtherTok{=}\NormalTok{yield\_t}
\NormalTok{SPR\_t}\OtherTok{=}\NormalTok{yield\_t}

\NormalTok{N\_a[}\DecValTok{1}\NormalTok{,]}\OtherTok{=}\NormalTok{R0}\SpecialCharTok{*}\NormalTok{survship0}
\NormalTok{F\_a[}\DecValTok{1}\NormalTok{,]}\OtherTok{=}\NormalTok{F\_full}\SpecialCharTok{*}\NormalTok{vuln}
\NormalTok{Z\_a[}\DecValTok{1}\NormalTok{,]}\OtherTok{=}\NormalTok{F\_a[}\DecValTok{1}\NormalTok{,]}\SpecialCharTok{+}\NormalTok{M}
\NormalTok{U\_a[}\DecValTok{1}\NormalTok{,]}\OtherTok{=}\DecValTok{1}\SpecialCharTok{{-}}\FunctionTok{exp}\NormalTok{(}\SpecialCharTok{{-}}\NormalTok{F\_a[}\DecValTok{1}\NormalTok{,])}
\NormalTok{yield\_t[}\DecValTok{1}\NormalTok{]}\OtherTok{=}\FunctionTok{sum}\NormalTok{(N\_a[}\DecValTok{1}\NormalTok{,]}\SpecialCharTok{*}\NormalTok{weight}\SpecialCharTok{*}\NormalTok{U\_a[}\DecValTok{1}\NormalTok{,])}
\NormalTok{ssb\_t[}\DecValTok{1}\NormalTok{]}\OtherTok{=}\FunctionTok{sum}\NormalTok{(N\_a[}\DecValTok{1}\NormalTok{,]}\SpecialCharTok{*}\NormalTok{fecundity)}
\NormalTok{SPR\_t[}\DecValTok{1}\NormalTok{]}\OtherTok{=}\NormalTok{ssb\_t[}\DecValTok{1}\NormalTok{]}\SpecialCharTok{/}\NormalTok{(R0}\SpecialCharTok{*}\NormalTok{ssb0)}

\ControlFlowTok{for}\NormalTok{(t }\ControlFlowTok{in} \DecValTok{2}\SpecialCharTok{:}\NormalTok{tmax)\{}
\NormalTok{  N\_a[t,}\DecValTok{1}\NormalTok{]}\OtherTok{=}\NormalTok{alpha}\SpecialCharTok{*}\NormalTok{ssb\_t[t}\DecValTok{{-}1}\NormalTok{]}\SpecialCharTok{*}\FunctionTok{exp}\NormalTok{(}\SpecialCharTok{{-}}\NormalTok{beta}\SpecialCharTok{*}\NormalTok{ssb\_t[t}\DecValTok{{-}1}\NormalTok{])}
  \ControlFlowTok{for}\NormalTok{(a }\ControlFlowTok{in} \DecValTok{2}\SpecialCharTok{:}\NormalTok{n\_ages)\{}
\NormalTok{    N\_a[t,a]}\OtherTok{=}\NormalTok{N\_a[t}\DecValTok{{-}1}\NormalTok{,a}\DecValTok{{-}1}\NormalTok{]}\SpecialCharTok{*}\FunctionTok{exp}\NormalTok{(}\SpecialCharTok{{-}}\NormalTok{Z\_a[t}\DecValTok{{-}1}\NormalTok{,a}\DecValTok{{-}1}\NormalTok{])}
\NormalTok{  \}}
\NormalTok{  N\_a[t,n\_ages]}\OtherTok{=}\NormalTok{N\_a[t,n\_ages]}\SpecialCharTok{+}\NormalTok{N\_a[t}\DecValTok{{-}1}\NormalTok{,a]}\SpecialCharTok{*}\FunctionTok{exp}\NormalTok{(}\SpecialCharTok{{-}}\NormalTok{Z\_a[t}\DecValTok{{-}1}\NormalTok{,a])  }
\NormalTok{  F\_a[t,]}\OtherTok{=}\NormalTok{F\_full}\SpecialCharTok{*}\NormalTok{vuln}
\NormalTok{  Z\_a[t,]}\OtherTok{=}\NormalTok{F\_a[t,]}\SpecialCharTok{+}\NormalTok{M}
\NormalTok{  U\_a[t,]}\OtherTok{=}\DecValTok{1}\SpecialCharTok{{-}}\FunctionTok{exp}\NormalTok{(}\SpecialCharTok{{-}}\NormalTok{F\_a[t,])}
\NormalTok{  yield\_t[t]}\OtherTok{=}\FunctionTok{sum}\NormalTok{(N\_a[t,]}\SpecialCharTok{*}\NormalTok{weight}\SpecialCharTok{*}\NormalTok{U\_a[t,])}
\NormalTok{  ssb\_t[t]}\OtherTok{=}\FunctionTok{sum}\NormalTok{(N\_a[t,]}\SpecialCharTok{*}\NormalTok{fecundity)}
\NormalTok{  SPR\_t[t]}\OtherTok{=}\NormalTok{ssb\_t[t]}\SpecialCharTok{/}\NormalTok{(R0}\SpecialCharTok{*}\NormalTok{ssb0)}
\NormalTok{\}}
  
\FunctionTok{windows}\NormalTok{()}
\FunctionTok{plot}\NormalTok{(SPR\_t,}\AttributeTok{type=}\StringTok{\textquotesingle{}l\textquotesingle{}}\NormalTok{,}\AttributeTok{ylim=}\FunctionTok{c}\NormalTok{(}\DecValTok{0}\NormalTok{,}\DecValTok{1}\NormalTok{)) }
\end{Highlighting}
\end{Shaded}

\subsection{Question Three}\label{question-three}

Read in the Cities.csv file from Canvas using a relative file path.

\begin{Shaded}
\begin{Highlighting}[]
\NormalTok{CITIES }\OtherTok{\textless{}{-}} \FunctionTok{read.csv}\NormalTok{(}\StringTok{"Cities.csv"}\NormalTok{)}
\FunctionTok{head}\NormalTok{(CITIES)}
\end{Highlighting}
\end{Shaded}

\begin{verbatim}
##          city  city_ascii state_id state_name county_fips county_name     lat
## 1    New York    New York       NY   New York       36081      Queens 40.6943
## 2 Los Angeles Los Angeles       CA California        6037 Los Angeles 34.1141
## 3     Chicago     Chicago       IL   Illinois       17031        Cook 41.8375
## 4       Miami       Miami       FL    Florida       12086  Miami-Dade 25.7840
## 5     Houston     Houston       TX      Texas       48201      Harris 29.7860
## 6      Dallas      Dallas       TX      Texas       48113      Dallas 32.7935
##        long population density
## 1  -73.9249   18832416 10943.7
## 2 -118.4068   11885717  3165.8
## 3  -87.6866    8489066  4590.3
## 4  -80.2101    6113982  4791.1
## 5  -95.3885    6046392  1386.5
## 6  -96.7667    5843632  1477.2
\end{verbatim}

\subsection{Question Four}\label{question-four}

Write a function to calculate the distance between two pairs of
coordinates based on the Haversine formula (see below). The input into
the function should be lat1, lon1, lat2, and lon2. The function should
return the object distance\_km. All the code below needs to go into the
function.

\#GIVEN CODE --\textgreater{}

\#Convert to radians rad.lat1 \textless- lat1 * pi/180 rad.lon1
\textless- lon1 * pi/180 rad.lat2 \textless- lat2 * pi/180 rad.lon2
\textless- lon2 * pi/180

\#Haversine Formula delta\_lat \textless- rad.lat2 - rad.lat1 delta\_lon
\textless- rad.lon2 - rad.lon1 a \textless- sin(delta\_lat / 2)\^{}2 +
cos(rad.lat1) * cos(rad.lat2) * sin(delta\_lon / 2)\^{}2 c \textless- 2
* asin(sqrt(a))

earth\_radius \textless- 6378137 \#Earth's radius in kilometers

distance\_km \textless- (earth\_radius * c)/1000 \#Calculate the
distance

\begin{Shaded}
\begin{Highlighting}[]
\CommentTok{\#Function Code}

\NormalTok{Calculate\_Distance }\OtherTok{\textless{}{-}}\ControlFlowTok{function}\NormalTok{ (lat1, lon1, lat2, lon2)\{}

\NormalTok{rad.lat1 }\OtherTok{\textless{}{-}}\NormalTok{ lat1 }\SpecialCharTok{*}\NormalTok{ pi}\SpecialCharTok{/}\DecValTok{180}
\NormalTok{rad.lon1 }\OtherTok{\textless{}{-}}\NormalTok{ lon1 }\SpecialCharTok{*}\NormalTok{ pi}\SpecialCharTok{/}\DecValTok{180}
\NormalTok{rad.lat2 }\OtherTok{\textless{}{-}}\NormalTok{ lat2 }\SpecialCharTok{*}\NormalTok{ pi}\SpecialCharTok{/}\DecValTok{180}
\NormalTok{rad.lon2 }\OtherTok{\textless{}{-}}\NormalTok{ lon2 }\SpecialCharTok{*}\NormalTok{ pi}\SpecialCharTok{/}\DecValTok{180}

\NormalTok{delta\_lat }\OtherTok{\textless{}{-}}\NormalTok{ rad.lat2 }\SpecialCharTok{{-}}\NormalTok{ rad.lat1}
\NormalTok{delta\_lon }\OtherTok{\textless{}{-}}\NormalTok{ rad.lon2 }\SpecialCharTok{{-}}\NormalTok{ rad.lon1}
\NormalTok{a }\OtherTok{\textless{}{-}} \FunctionTok{sin}\NormalTok{(delta\_lat }\SpecialCharTok{/} \DecValTok{2}\NormalTok{)}\SpecialCharTok{\^{}}\DecValTok{2} \SpecialCharTok{+} \FunctionTok{cos}\NormalTok{(rad.lat1) }\SpecialCharTok{*} \FunctionTok{cos}\NormalTok{(rad.lat2) }\SpecialCharTok{*} \FunctionTok{sin}\NormalTok{(delta\_lon }\SpecialCharTok{/} \DecValTok{2}\NormalTok{)}\SpecialCharTok{\^{}}\DecValTok{2}

\NormalTok{c }\OtherTok{\textless{}{-}} \DecValTok{2} \SpecialCharTok{*} \FunctionTok{asin}\NormalTok{(}\FunctionTok{sqrt}\NormalTok{(a)) }

\NormalTok{Earth\_Radius }\OtherTok{\textless{}{-}} \DecValTok{6378137}

\NormalTok{distance\_km }\OtherTok{\textless{}{-}}\NormalTok{ (Earth\_Radius }\SpecialCharTok{*}\NormalTok{ c)}\SpecialCharTok{/}\DecValTok{1000}
\FunctionTok{return}\NormalTok{ (distance\_km)}
\NormalTok{\}}
\end{Highlighting}
\end{Shaded}

\subsection{Question Five}\label{question-five}

Using your function, compute the distance between Auburn, AL and New
York City a. Subset/filter the Cities.csv data to include only the
latitude and longitude values you need and input as input to your
function. b. The output of your function should be 1367.854 km

\begin{Shaded}
\begin{Highlighting}[]
\NormalTok{AU\_lat}\OtherTok{=}\NormalTok{CITIES}\SpecialCharTok{$}\NormalTok{lat[CITIES}\SpecialCharTok{$}\NormalTok{city}\SpecialCharTok{==}\StringTok{"Auburn"}\NormalTok{]}
\NormalTok{AU\_lon}\OtherTok{=}\NormalTok{CITIES}\SpecialCharTok{$}\NormalTok{long[CITIES}\SpecialCharTok{$}\NormalTok{city}\SpecialCharTok{==}\StringTok{"Auburn"}\NormalTok{]}
\NormalTok{NY\_lat}\OtherTok{=}\NormalTok{CITIES}\SpecialCharTok{$}\NormalTok{lat[CITIES}\SpecialCharTok{$}\NormalTok{city}\SpecialCharTok{==}\StringTok{"New York"}\NormalTok{]}
\NormalTok{NY\_lon}\OtherTok{=}\NormalTok{CITIES}\SpecialCharTok{$}\NormalTok{long[CITIES}\SpecialCharTok{$}\NormalTok{city}\SpecialCharTok{==}\StringTok{"New York"}\NormalTok{]}

\FunctionTok{Calculate\_Distance}\NormalTok{(AU\_lat,AU\_lon, NY\_lat, NY\_lon)}
\end{Highlighting}
\end{Shaded}

\begin{verbatim}
## [1] 1367.854
\end{verbatim}

\subsection{Question Six}\label{question-six}

Now, use your function within a for loop to calculate the distance
between all other cities in the data. The output of the first 9
iterations is shown below.

\begin{Shaded}
\begin{Highlighting}[]
\CommentTok{\# Calculate distances and add to csv file}

\NormalTok{VALUES}\OtherTok{\textless{}{-}} \FunctionTok{unique}\NormalTok{(CITIES}\SpecialCharTok{$}\NormalTok{city)}

\NormalTok{distances }\OtherTok{\textless{}{-}} \FunctionTok{numeric}\NormalTok{(}\FunctionTok{length}\NormalTok{(VALUES))}
 
 \ControlFlowTok{for}\NormalTok{ (i }\ControlFlowTok{in} \FunctionTok{seq\_along}\NormalTok{(VALUES)) \{}
\NormalTok{   lat1 }\OtherTok{\textless{}{-}}\NormalTok{ CITIES}\SpecialCharTok{$}\NormalTok{lat[CITIES}\SpecialCharTok{$}\NormalTok{city }\SpecialCharTok{==} \StringTok{"Auburn"}\NormalTok{]}
\NormalTok{   lon1 }\OtherTok{\textless{}{-}}\NormalTok{ CITIES}\SpecialCharTok{$}\NormalTok{lon[CITIES}\SpecialCharTok{$}\NormalTok{city }\SpecialCharTok{==} \StringTok{"Auburn"}\NormalTok{]}
\NormalTok{   lat2 }\OtherTok{\textless{}{-}}\NormalTok{ CITIES}\SpecialCharTok{$}\NormalTok{lat[CITIES}\SpecialCharTok{$}\NormalTok{city }\SpecialCharTok{==}\NormalTok{ VALUES[i]]}
\NormalTok{   lon2 }\OtherTok{\textless{}{-}}\NormalTok{ CITIES}\SpecialCharTok{$}\NormalTok{lon[CITIES}\SpecialCharTok{$}\NormalTok{city }\SpecialCharTok{==}\NormalTok{ VALUES[i]]}
\NormalTok{   distances[i] }\OtherTok{\textless{}{-}} \FunctionTok{Calculate\_Distance}\NormalTok{(lat1, lon1, lat2, lon2)}
\NormalTok{ \}}
 
\NormalTok{CITIES}\SpecialCharTok{$}\NormalTok{distance\_from\_Auburn }\OtherTok{\textless{}{-}}\NormalTok{ distances}
 
\FunctionTok{head}\NormalTok{(CITIES)}
\end{Highlighting}
\end{Shaded}

\begin{verbatim}
##          city  city_ascii state_id state_name county_fips county_name     lat
## 1    New York    New York       NY   New York       36081      Queens 40.6943
## 2 Los Angeles Los Angeles       CA California        6037 Los Angeles 34.1141
## 3     Chicago     Chicago       IL   Illinois       17031        Cook 41.8375
## 4       Miami       Miami       FL    Florida       12086  Miami-Dade 25.7840
## 5     Houston     Houston       TX      Texas       48201      Harris 29.7860
## 6      Dallas      Dallas       TX      Texas       48113      Dallas 32.7935
##        long population density distance_from_Auburn
## 1  -73.9249   18832416 10943.7            1367.8540
## 2 -118.4068   11885717  3165.8            3051.8382
## 3  -87.6866    8489066  4590.3            1045.5213
## 4  -80.2101    6113982  4791.1             916.4138
## 5  -95.3885    6046392  1386.5             993.0298
## 6  -96.7667    5843632  1477.2            1056.0217
\end{verbatim}

\subsubsection{Bonus Points}\label{bonus-points}

Bonus point if you can have the output of each iteration append a new
row to a dataframe, generating a new column of data. In other words, the
loop should create a dataframe with three columns called city1, city2,
and distance\_km, as shown below. The first six rows of the dataframe
are shown below.

\subsection{Question Seven}\label{question-seven}

Commit and push a gfm .md file to GitHub inside a directory called
Coding Challenge 6. Provide me a link to your github written as a
clickable link in your .pdf or .docx

\href{https://github.com/Mads-Hamrick/ENTM6820class/tree/9c303debe9565060bec454ea4d538b01e052f478/2025_4_4_CodingChallengeSix_mer0127}{Coding
Challenge Six}

\end{document}
