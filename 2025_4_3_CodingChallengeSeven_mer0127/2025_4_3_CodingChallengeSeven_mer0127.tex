% Options for packages loaded elsewhere
\PassOptionsToPackage{unicode}{hyperref}
\PassOptionsToPackage{hyphens}{url}
%
\documentclass[
  12pt,
]{article}
\usepackage{amsmath,amssymb}
\usepackage{iftex}
\ifPDFTeX
  \usepackage[T1]{fontenc}
  \usepackage[utf8]{inputenc}
  \usepackage{textcomp} % provide euro and other symbols
\else % if luatex or xetex
  \usepackage{unicode-math} % this also loads fontspec
  \defaultfontfeatures{Scale=MatchLowercase}
  \defaultfontfeatures[\rmfamily]{Ligatures=TeX,Scale=1}
\fi
\usepackage{lmodern}
\ifPDFTeX\else
  % xetex/luatex font selection
\fi
% Use upquote if available, for straight quotes in verbatim environments
\IfFileExists{upquote.sty}{\usepackage{upquote}}{}
\IfFileExists{microtype.sty}{% use microtype if available
  \usepackage[]{microtype}
  \UseMicrotypeSet[protrusion]{basicmath} % disable protrusion for tt fonts
}{}
\makeatletter
\@ifundefined{KOMAClassName}{% if non-KOMA class
  \IfFileExists{parskip.sty}{%
    \usepackage{parskip}
  }{% else
    \setlength{\parindent}{0pt}
    \setlength{\parskip}{6pt plus 2pt minus 1pt}}
}{% if KOMA class
  \KOMAoptions{parskip=half}}
\makeatother
\usepackage{xcolor}
\usepackage[margin=1in]{geometry}
\usepackage{color}
\usepackage{fancyvrb}
\newcommand{\VerbBar}{|}
\newcommand{\VERB}{\Verb[commandchars=\\\{\}]}
\DefineVerbatimEnvironment{Highlighting}{Verbatim}{commandchars=\\\{\}}
% Add ',fontsize=\small' for more characters per line
\usepackage{framed}
\definecolor{shadecolor}{RGB}{248,248,248}
\newenvironment{Shaded}{\begin{snugshade}}{\end{snugshade}}
\newcommand{\AlertTok}[1]{\textcolor[rgb]{0.94,0.16,0.16}{#1}}
\newcommand{\AnnotationTok}[1]{\textcolor[rgb]{0.56,0.35,0.01}{\textbf{\textit{#1}}}}
\newcommand{\AttributeTok}[1]{\textcolor[rgb]{0.13,0.29,0.53}{#1}}
\newcommand{\BaseNTok}[1]{\textcolor[rgb]{0.00,0.00,0.81}{#1}}
\newcommand{\BuiltInTok}[1]{#1}
\newcommand{\CharTok}[1]{\textcolor[rgb]{0.31,0.60,0.02}{#1}}
\newcommand{\CommentTok}[1]{\textcolor[rgb]{0.56,0.35,0.01}{\textit{#1}}}
\newcommand{\CommentVarTok}[1]{\textcolor[rgb]{0.56,0.35,0.01}{\textbf{\textit{#1}}}}
\newcommand{\ConstantTok}[1]{\textcolor[rgb]{0.56,0.35,0.01}{#1}}
\newcommand{\ControlFlowTok}[1]{\textcolor[rgb]{0.13,0.29,0.53}{\textbf{#1}}}
\newcommand{\DataTypeTok}[1]{\textcolor[rgb]{0.13,0.29,0.53}{#1}}
\newcommand{\DecValTok}[1]{\textcolor[rgb]{0.00,0.00,0.81}{#1}}
\newcommand{\DocumentationTok}[1]{\textcolor[rgb]{0.56,0.35,0.01}{\textbf{\textit{#1}}}}
\newcommand{\ErrorTok}[1]{\textcolor[rgb]{0.64,0.00,0.00}{\textbf{#1}}}
\newcommand{\ExtensionTok}[1]{#1}
\newcommand{\FloatTok}[1]{\textcolor[rgb]{0.00,0.00,0.81}{#1}}
\newcommand{\FunctionTok}[1]{\textcolor[rgb]{0.13,0.29,0.53}{\textbf{#1}}}
\newcommand{\ImportTok}[1]{#1}
\newcommand{\InformationTok}[1]{\textcolor[rgb]{0.56,0.35,0.01}{\textbf{\textit{#1}}}}
\newcommand{\KeywordTok}[1]{\textcolor[rgb]{0.13,0.29,0.53}{\textbf{#1}}}
\newcommand{\NormalTok}[1]{#1}
\newcommand{\OperatorTok}[1]{\textcolor[rgb]{0.81,0.36,0.00}{\textbf{#1}}}
\newcommand{\OtherTok}[1]{\textcolor[rgb]{0.56,0.35,0.01}{#1}}
\newcommand{\PreprocessorTok}[1]{\textcolor[rgb]{0.56,0.35,0.01}{\textit{#1}}}
\newcommand{\RegionMarkerTok}[1]{#1}
\newcommand{\SpecialCharTok}[1]{\textcolor[rgb]{0.81,0.36,0.00}{\textbf{#1}}}
\newcommand{\SpecialStringTok}[1]{\textcolor[rgb]{0.31,0.60,0.02}{#1}}
\newcommand{\StringTok}[1]{\textcolor[rgb]{0.31,0.60,0.02}{#1}}
\newcommand{\VariableTok}[1]{\textcolor[rgb]{0.00,0.00,0.00}{#1}}
\newcommand{\VerbatimStringTok}[1]{\textcolor[rgb]{0.31,0.60,0.02}{#1}}
\newcommand{\WarningTok}[1]{\textcolor[rgb]{0.56,0.35,0.01}{\textbf{\textit{#1}}}}
\usepackage{graphicx}
\makeatletter
\def\maxwidth{\ifdim\Gin@nat@width>\linewidth\linewidth\else\Gin@nat@width\fi}
\def\maxheight{\ifdim\Gin@nat@height>\textheight\textheight\else\Gin@nat@height\fi}
\makeatother
% Scale images if necessary, so that they will not overflow the page
% margins by default, and it is still possible to overwrite the defaults
% using explicit options in \includegraphics[width, height, ...]{}
\setkeys{Gin}{width=\maxwidth,height=\maxheight,keepaspectratio}
% Set default figure placement to htbp
\makeatletter
\def\fps@figure{htbp}
\makeatother
\setlength{\emergencystretch}{3em} % prevent overfull lines
\providecommand{\tightlist}{%
  \setlength{\itemsep}{0pt}\setlength{\parskip}{0pt}}
\setcounter{secnumdepth}{-\maxdimen} % remove section numbering
\ifLuaTeX
  \usepackage{selnolig}  % disable illegal ligatures
\fi
\usepackage{bookmark}
\IfFileExists{xurl.sty}{\usepackage{xurl}}{} % add URL line breaks if available
\urlstyle{same}
\hypersetup{
  pdftitle={2025\_4\_3\_CodingChallengeSeven\_mer0127},
  pdfauthor={Madeline Redd},
  hidelinks,
  pdfcreator={LaTeX via pandoc}}

\title{2025\_4\_3\_CodingChallengeSeven\_mer0127}
\author{Madeline Redd}
\date{2025 - 04 - 03}

\begin{document}
\maketitle

\subsection{Question One}\label{question-one}

Read in the data called ``PlantEmergence.csv'' using a relative file
path and load the following libraries. tidyverse, lme4, emmeans,
multcomp, and multcompView. Turn the Treatment , DaysAfterPlanting and
Rep into factors using the function as.factor

\begin{Shaded}
\begin{Highlighting}[]
\FunctionTok{library}\NormalTok{(tidyverse)}
\end{Highlighting}
\end{Shaded}

\begin{verbatim}
## -- Attaching core tidyverse packages ------------------------ tidyverse 2.0.0 --
## v dplyr     1.1.4     v readr     2.1.5
## v forcats   1.0.0     v stringr   1.5.1
## v ggplot2   3.5.1     v tibble    3.2.1
## v lubridate 1.9.4     v tidyr     1.3.1
## v purrr     1.0.2     
## -- Conflicts ------------------------------------------ tidyverse_conflicts() --
## x dplyr::filter() masks stats::filter()
## x dplyr::lag()    masks stats::lag()
## i Use the conflicted package (<http://conflicted.r-lib.org/>) to force all conflicts to become errors
\end{verbatim}

\begin{Shaded}
\begin{Highlighting}[]
\FunctionTok{library}\NormalTok{(lme4)}
\end{Highlighting}
\end{Shaded}

\begin{verbatim}
## Loading required package: Matrix
## 
## Attaching package: 'Matrix'
## 
## The following objects are masked from 'package:tidyr':
## 
##     expand, pack, unpack
\end{verbatim}

\begin{Shaded}
\begin{Highlighting}[]
\FunctionTok{library}\NormalTok{(emmeans)}
\end{Highlighting}
\end{Shaded}

\begin{verbatim}
## Welcome to emmeans.
## Caution: You lose important information if you filter this package's results.
## See '? untidy'
\end{verbatim}

\begin{Shaded}
\begin{Highlighting}[]
\FunctionTok{library}\NormalTok{(multcompView)}
\FunctionTok{library}\NormalTok{(multcomp)}
\end{Highlighting}
\end{Shaded}

\begin{verbatim}
## Loading required package: mvtnorm
## Loading required package: survival
## Loading required package: TH.data
## Loading required package: MASS
## 
## Attaching package: 'MASS'
## 
## The following object is masked from 'package:dplyr':
## 
##     select
## 
## 
## Attaching package: 'TH.data'
## 
## The following object is masked from 'package:MASS':
## 
##     geyser
\end{verbatim}

\begin{Shaded}
\begin{Highlighting}[]
\NormalTok{PlantEmerg }\OtherTok{\textless{}{-}} \FunctionTok{read.csv}\NormalTok{(}\StringTok{"PlantEmergence.csv"}\NormalTok{)}

\FunctionTok{head}\NormalTok{(PlantEmerg)}
\end{Highlighting}
\end{Shaded}

\begin{verbatim}
##   Plot Treatment Rep Emergence DatePlanted DateCounted DaysAfterPlanting
## 1  101         1   1     180.5    9-May-22   16-May-22                 7
## 2  102         2   1      54.5    9-May-22   16-May-22                 7
## 3  103         3   1     195.0    9-May-22   16-May-22                 7
## 4  104         4   1     198.5    9-May-22   16-May-22                 7
## 5  105         5   1     202.0    9-May-22   16-May-22                 7
## 6  106         6   1     184.0    9-May-22   16-May-22                 7
\end{verbatim}

\subsection{Question Two}\label{question-two}

Fit a linear model to predict Emergence using Treatment and
DaysAfterPlanting along with the interaction. Provide the summary of the
linear model and ANOVA results.

\begin{Shaded}
\begin{Highlighting}[]
\NormalTok{lm\_Q2}\OtherTok{\textless{}{-}} \FunctionTok{lm}\NormalTok{(DaysAfterPlanting }\SpecialCharTok{\textasciitilde{}}\NormalTok{ Emergence, }\AttributeTok{data =}\NormalTok{ PlantEmerg)}

\FunctionTok{summary}\NormalTok{(lm\_Q2)}
\end{Highlighting}
\end{Shaded}

\begin{verbatim}
## 
## Call:
## lm(formula = DaysAfterPlanting ~ Emergence, data = PlantEmerg)
## 
## Residuals:
##      Min       1Q   Median       3Q      Max 
## -10.9250  -5.1621   0.7038   6.6266  12.5531 
## 
## Coefficients:
##             Estimate Std. Error t value Pr(>|t|)    
## (Intercept) 14.85867    2.67842   5.548 1.37e-07 ***
## Emergence    0.01471    0.01446   1.017    0.311    
## ---
## Signif. codes:  0 '***' 0.001 '**' 0.01 '*' 0.05 '.' 0.1 ' ' 1
## 
## Residual standard error: 7.853 on 142 degrees of freedom
## Multiple R-squared:  0.007231,   Adjusted R-squared:  0.0002393 
## F-statistic: 1.034 on 1 and 142 DF,  p-value: 0.3109
\end{verbatim}

\subsection{Question Three}\label{question-three}

Based on the results of the linear model in question 2, do you need to
fit the interaction term? Provide a simplified linear model without the
interaction term but still testing both main effects. Provide the
summary and ANOVA results. Then, interpret the intercept and the
coefficient for Treatment 2.

\begin{Shaded}
\begin{Highlighting}[]
\NormalTok{lm\_Q2}\OtherTok{\textless{}{-}} \FunctionTok{lm}\NormalTok{(DaysAfterPlanting }\SpecialCharTok{\textasciitilde{}}\NormalTok{ Emergence, }\AttributeTok{data =}\NormalTok{ PlantEmerg)}

\FunctionTok{summary}\NormalTok{(lm\_Q2)}
\end{Highlighting}
\end{Shaded}

\begin{verbatim}
## 
## Call:
## lm(formula = DaysAfterPlanting ~ Emergence, data = PlantEmerg)
## 
## Residuals:
##      Min       1Q   Median       3Q      Max 
## -10.9250  -5.1621   0.7038   6.6266  12.5531 
## 
## Coefficients:
##             Estimate Std. Error t value Pr(>|t|)    
## (Intercept) 14.85867    2.67842   5.548 1.37e-07 ***
## Emergence    0.01471    0.01446   1.017    0.311    
## ---
## Signif. codes:  0 '***' 0.001 '**' 0.01 '*' 0.05 '.' 0.1 ' ' 1
## 
## Residual standard error: 7.853 on 142 degrees of freedom
## Multiple R-squared:  0.007231,   Adjusted R-squared:  0.0002393 
## F-statistic: 1.034 on 1 and 142 DF,  p-value: 0.3109
\end{verbatim}

\subsection{Question Four}\label{question-four}

Calculate the least square means for Treatment using the emmeans package
and perform a Tukey separation with the compact letter display using the
cld function. Interpret the results.

\begin{Shaded}
\begin{Highlighting}[]
\NormalTok{lm\_Q2}\OtherTok{\textless{}{-}} \FunctionTok{lm}\NormalTok{(DaysAfterPlanting }\SpecialCharTok{\textasciitilde{}}\NormalTok{ Emergence, }\AttributeTok{data =}\NormalTok{ PlantEmerg)}

\FunctionTok{summary}\NormalTok{(lm\_Q2)}
\end{Highlighting}
\end{Shaded}

\begin{verbatim}
## 
## Call:
## lm(formula = DaysAfterPlanting ~ Emergence, data = PlantEmerg)
## 
## Residuals:
##      Min       1Q   Median       3Q      Max 
## -10.9250  -5.1621   0.7038   6.6266  12.5531 
## 
## Coefficients:
##             Estimate Std. Error t value Pr(>|t|)    
## (Intercept) 14.85867    2.67842   5.548 1.37e-07 ***
## Emergence    0.01471    0.01446   1.017    0.311    
## ---
## Signif. codes:  0 '***' 0.001 '**' 0.01 '*' 0.05 '.' 0.1 ' ' 1
## 
## Residual standard error: 7.853 on 142 degrees of freedom
## Multiple R-squared:  0.007231,   Adjusted R-squared:  0.0002393 
## F-statistic: 1.034 on 1 and 142 DF,  p-value: 0.3109
\end{verbatim}

\subsection{Question Five}\label{question-five}

The provided function lets you dynamically add a linear model plus one
factor from that model and plots a bar chart with letters denoting
treatment differences. Use this model to generate the plot shown below.
Explain the significance of the letters.

\begin{Shaded}
\begin{Highlighting}[]
\NormalTok{plot\_cldbars\_onefactor }\OtherTok{\textless{}{-}} \ControlFlowTok{function}\NormalTok{(lm\_model, factor) \{}
\NormalTok{  data }\OtherTok{\textless{}{-}}\NormalTok{ lm\_model}\SpecialCharTok{$}\NormalTok{model}
\NormalTok{  variables }\OtherTok{\textless{}{-}} \FunctionTok{colnames}\NormalTok{(lm\_model}\SpecialCharTok{$}\NormalTok{model)}
\NormalTok{  dependent\_var }\OtherTok{\textless{}{-}}\NormalTok{ variables[}\DecValTok{1}\NormalTok{]}
\NormalTok{  independent\_var }\OtherTok{\textless{}{-}}\NormalTok{ variables[}\DecValTok{2}\SpecialCharTok{:}\FunctionTok{length}\NormalTok{(variables)]}

\NormalTok{  lsmeans }\OtherTok{\textless{}{-}} \FunctionTok{emmeans}\NormalTok{(lm\_model, }\FunctionTok{as.formula}\NormalTok{(}\FunctionTok{paste}\NormalTok{(}\StringTok{"\textasciitilde{}"}\NormalTok{, factor))) }\CommentTok{\#Estimate lsmeans }
\NormalTok{  Results\_lsmeans }\OtherTok{\textless{}{-}} \FunctionTok{cld}\NormalTok{(lsmeans, }\AttributeTok{alpha =} \FloatTok{0.05}\NormalTok{, }\AttributeTok{reversed =} \ConstantTok{TRUE}\NormalTok{, }\AttributeTok{details =} \ConstantTok{TRUE}\NormalTok{, }
                \AttributeTok{Letters =}\NormalTok{ letters) }\CommentTok{\# contrast with Tukey adjustment by default.}
  
    \CommentTok{\# Extracting the letters for the bars}
\NormalTok{  sig.diff.letters }\OtherTok{\textless{}{-}} \FunctionTok{data.frame}\NormalTok{(Results\_lsmeans}\SpecialCharTok{$}\NormalTok{emmeans[,}\DecValTok{1}\NormalTok{], }
                                 \FunctionTok{str\_trim}\NormalTok{(Results\_lsmeans}\SpecialCharTok{$}\NormalTok{emmeans[,}\DecValTok{7}\NormalTok{]))}
  \FunctionTok{colnames}\NormalTok{(sig.diff.letters) }\OtherTok{\textless{}{-}} \FunctionTok{c}\NormalTok{(factor, }\StringTok{"Letters"}\NormalTok{)}
  
  \CommentTok{\# for plotting with letters from significance test}
\NormalTok{  ave\_stand2 }\OtherTok{\textless{}{-}}\NormalTok{ lm\_model}\SpecialCharTok{$}\NormalTok{model }\SpecialCharTok{\%\textgreater{}\%}
    \FunctionTok{group\_by}\NormalTok{(}\SpecialCharTok{!!}\FunctionTok{sym}\NormalTok{(factor)) }\SpecialCharTok{\%\textgreater{}\%}
\NormalTok{    dplyr}\SpecialCharTok{::}\FunctionTok{summarize}\NormalTok{(}
      \AttributeTok{ave.emerge =} \FunctionTok{mean}\NormalTok{(.data[[dependent\_var]], }\AttributeTok{na.rm =} \ConstantTok{TRUE}\NormalTok{),}
      \AttributeTok{se =} \FunctionTok{sd}\NormalTok{(.data[[dependent\_var]]) }\SpecialCharTok{/} \FunctionTok{sqrt}\NormalTok{(}\FunctionTok{n}\NormalTok{())}
\NormalTok{    ) }\SpecialCharTok{\%\textgreater{}\%}
    \FunctionTok{left\_join}\NormalTok{(sig.diff.letters, }\AttributeTok{by =}\NormalTok{ factor) }\SpecialCharTok{\%\textgreater{}\%}
    \FunctionTok{mutate}\NormalTok{(}\AttributeTok{letter\_position =}\NormalTok{ ave.emerge }\SpecialCharTok{+} \DecValTok{10} \SpecialCharTok{*}\NormalTok{ se)}
  
\NormalTok{  plot }\OtherTok{\textless{}{-}} \FunctionTok{ggplot}\NormalTok{(data, }\FunctionTok{aes}\NormalTok{(}\AttributeTok{x =} \SpecialCharTok{!!} \FunctionTok{sym}\NormalTok{(factor), }\AttributeTok{y =} \SpecialCharTok{!!} \FunctionTok{sym}\NormalTok{(dependent\_var))) }\SpecialCharTok{+} 
    \FunctionTok{stat\_summary}\NormalTok{(}\AttributeTok{fun =}\NormalTok{ mean, }\AttributeTok{geom =} \StringTok{"bar"}\NormalTok{) }\SpecialCharTok{+}
    \FunctionTok{stat\_summary}\NormalTok{(}\AttributeTok{fun.data =}\NormalTok{ mean\_se, }\AttributeTok{geom =} \StringTok{"errorbar"}\NormalTok{, }\AttributeTok{width =} \FloatTok{0.5}\NormalTok{) }\SpecialCharTok{+}
    \FunctionTok{ylab}\NormalTok{(}\StringTok{"Number of emerged plants"}\NormalTok{) }\SpecialCharTok{+} 
    \FunctionTok{geom\_jitter}\NormalTok{(}\AttributeTok{width =} \FloatTok{0.02}\NormalTok{, }\AttributeTok{alpha =} \FloatTok{0.5}\NormalTok{) }\SpecialCharTok{+}
    \FunctionTok{geom\_text}\NormalTok{(}\AttributeTok{data =}\NormalTok{ ave\_stand2, }\FunctionTok{aes}\NormalTok{(}\AttributeTok{label =}\NormalTok{ Letters, }\AttributeTok{y =}\NormalTok{ letter\_position), }
              \AttributeTok{size =} \DecValTok{5}\NormalTok{) }\SpecialCharTok{+}
    \FunctionTok{xlab}\NormalTok{(}\FunctionTok{as.character}\NormalTok{(factor)) }\SpecialCharTok{+}
    \FunctionTok{theme\_classic}\NormalTok{()}
  
  \FunctionTok{return}\NormalTok{(plot)}
\NormalTok{\}}
\end{Highlighting}
\end{Shaded}

\subsection{Question Six}\label{question-six}

Generate the gfm .md file along with a .html, .docx, or .pdf. Commit,
and push the .md file to github and turn in the .html, .docx, or .pdf to
Canvas. Provide me a link here to your github.

{[}Coding Challenge Seven Link{]} ()

\end{document}
